%check the reads coverage in the sample where it is from
%
%\section{de bruijn graph mapping}
%
%check the reads coverage in another sample
%
%\section{IGS-based diversity analysis}
%


\chapter{Conclusion}

%Advantage:

%not only HMP, high depth, data
%but also low depth data, like soil metagenomics, which is impossible to use traditional method 


We have developed a novel statistical framework to enable microbial diversity
 analysis using whole genome shotgun metagenomic reads data without the
requirement of assembly, binning, reference or annotation. This dissertation
covers an overview of existing approaches of microbial diversity analysis
 of metagenomic samples, especially based on the concept of OTU, including the
steps in the procedure, such as contigs binning, statistical
analysis of OTU abundance information to estimate the microbial diversity. Next
the statistical framework based on the novel concept of IGS was discussed. As
the foundation of the framework, we described a novel method to count k-mers
efficiently and a scalable approach to retrieve the coverage of a read in a 
data set based on efficient and online k-mer counting. We also introduced the
applications of this approach in reducing the redundancy of metagenomic reads
dataset and analyzing sequencing error, which is beneficial to other tasks 
in metagenomic data analysis, like
assembly or error trimming. Finally, we discussed how we developed the concept 
of IGS based on the methods of efficient k-mer counting and digital
normalization discussed before.  
The application of IGS to analyze microbial diversity of metagenomic data sets 
was discussed and the performance of the IGS method on simulated data sets and
real data sets were demonstrated in the final chapter. In this chapter, we
summarize how the novel statistical framework based on IGS makes a difference to
 the diversity analysis in current microbial ecology research. 
Finally some directions of future work will be discussed.

  

\section{Summary}

Diversity analysis is a key part of the microbial ecology research, like of
macroorganism ecology. However due to the obscure definition of the term
"species" in microbial ecology, we can virtually never measure the diversity of
species directly, rather we use other taxonomic concepts like operational
taxonomic unit (OTU) to evaluate the diversity of microbial community, instead
of species. 16S rRNA sequencing reads may be classified into different OTUs.
Shotgun whole genome sequencing reads can also be classified into OTUs. But
most, if not all the existing methods based on the concept of OTU rely heavily on  
preprocessing of original reads data in some way like assembly or external 
information like reference sequences for annotation. Both of the prerequisites
are not satisfied for many metagenomic projects. For metagenomic data set with
low sequencing coverage, the assembly process skews the analysis by including 
primarily the most abundant organisms. Sequences that are rare are not 
assembled into contigs and are therefore not included in the contig analysis
It is common that only a small proportion of
reads can be used in assembly especially in a complex environmental sample. 
\cite{Howe2012} The reference sequence database is far from
completion especially for microbes in environmental samples from soil or sea water.
The paucity of reference databases affects the ability to identify functional 
capacity of microbes, when traits cannot be identified. 
Thus, applying these methods can only obtain an incomplete diversity of
the microbial community in the metagenomic sequencing data set. 

The IGS based
method discussed in this dissertation offers a novel framework that
overcomes the limitations of assembly, binning, or annotation without the
requirement of reference sequence database. It can take advantage of all the 
information in the metagenomic reads data and gain a full picture of the
diversity of the microbial community. Importantly, this is a new
framework with the concept of IGS instead of OTU as the taxonomic unit to
analyze microbial diversity. Thus, this framework can be used to perform all possible
diversity analysis that OTU-based framework can do. Moreover, this is a more thorough
approach than many other methods developed to solve only specific
problems in the field of diversity analysis. For example, there are several
methods developed to estimate the species richness in a metagenomic sample 
\cite{Rodriguez-R2014}. Our IGS based framework cannot only
estimate the species richness or size of metagenome as shown in the section above, but
also it can estimate the evenness or species abundance distribution of a
metagenomic sample, which is also an important aspect of alpha diversity
analysis. For beta diversity or compositional similarity analysis between
metagenomic samples, there are several methods developed to compare metagenomic
samples based on reads mapping or counting shared reads \cite{Rodriguez-R:2013aa}.
However they only estimate abundance-based similarity, similar to
the Bray-Curtis indices used in the experiment discussed in the section above.
It should be noted that the 
IGS-based framework can also be used to estimate incidence-based similarity, 
which cannot be estimated using other existing approaches.


Besides the potential for a broad application of the IGS based framework, 
it is also efficient and
highly scalable to handle extremely large metagenomic sequencing data
sets. We have discussed the efficiency of the novel k-mer counting method and
the following method of digital normalization, with the ability to retrieve the
coverage of a read accurately and efficiently. We also performed a thorough analysis
to examine the effect of the size of used data structure to the accuracy. We can take
advantage of the probabilistic characteristics of the data structure to make a
trade-off between expected analysis accuracy and expected usage of
computational power. In this way, we make the analysis highly scalable to keep
pace with the increasing size of metagenomic sequencing data.  

In addition, we examined the effect of sequencing depth to the accuracy of 
estimating microbial diversity.
It was expected that using more number of reads, that is, a data set with higher sequencing
depth increases the accuracy of diversity estimation. For similarity analysis 
between samples (beta diversity), a data set with relatively
low sequencing depth can still get decent results, as shown in the experiment
with synthetic data and real soil data sets. However, for alpha diversity such as richness
estimation, use of a data set with lower sequencing depth results in the diversity estimation
more distant from the real number. Although the absolute value of such species
richness of a sample is not accurate, the relative comparison of species
richness between samples is less prone to smaller reads data with lower
sequencing depth. These results suggested that for a specific
purpose, only a subset of the large metagenomic reads data can be enough to
achieve reasonably satisfying result. Under certain circumstance, this feature
is quite helpful and can reduce the computational expense dramatically. 


\section{Future work}


Though this dissertation demonstrated the performance of the new approach to 
analyze microbial diversity
using whole genome shotgun sequencing data without the requirement of assembly,
binning, or annotation, there is still plenty of room for improvement.  

Primary questions in the process of developing the IGS based method are how many
species there are in a sample or how similar the samples are with each
other, mostly focusing on the quantitative aspect. Admittedly these are 
important questions to the microbial ecologists, but they are also
curious about the qualitative aspect, such as what drives differences between samples
and eventually its functional potential\cite{Xu2014}.  Thus, a natural 
expansion of the IGS based framework will focus on
answering questions mentioned above. 

Now we have an efficient and scalable approach to obtain the coverage of a read in
a sample, it is straightforward to extract the reads according to its coverage
profile across samples so we can get a subset of reads that have specific
properties, like the reads that are common in all samples. In this way we
may collect these "common" reads across the samples and try to co-assemble
them since now they should have higher coverage. Or we can get a subset of
reads that are common in a group of samples but do not exist in another group
of samples, like the samples from patients and healthy persons. These
"signature" reads may offer important insights to understand what happens
to the microbial community while the environment changes. Admittedly these
kinds of "extraction" can be implemented using other methods like reads 
alignment method. However, they may not be as efficient and scalable as the IGS
based method, especially for extremely large metagenomic data.


One advantage of the IGS based method is that binning is not required in this
procedure. Firstly, traditionally binning is used to classify contigs after read
assembly effort. The similarity based binning method relies on
sequence alignment, which is inefficient, even infeasible for large
metagenomic data. Secondly, reference sequences are normally required for
similarity based binning approach. The composition-based approach relies on the
frequency profile of sequence signatures and machine learning approach on that
profile, which is computationally expensive. 
The third approach based on coverage profile across samples was developed
recently\cite{Albertsen2013,Karlsson2013,Alneberg2014,Nielsen2014,Imelfort2014}.Mostly, the coverage
profile is used with the companion of composition frequency profile to classify
contigs. The assumption on which the coverage profile based binning approaches
are based on, that contigs with similar coverage profile across samples are
more likely to be from the same microbial species, is actually similar to the
assumption on which using IGS to do beta diversity is based, that the IGSs with
similar coverage profile across samples are likely to be from the same
microbial organism. Thus, it is promising to classify the IGSs by the coverage
profile across samples. We have already overcome the challenge of retrieving the
coverage profile efficiently based on the probabilistic data structure, while
in those coverage profile based binning approaches the coverage profile is 
normally retrieved by assembly of contigs and mapping
reads back to contigs, which both require higher coverage reads to do assembly
and are computationally expensive. 

There are two obstacles to overcome in this coverage profile based
IGS binning approach. First, with relatively small number of samples, the
resolution will be limited, since the total number of different coverage
profiles will be limited. This is probably the reason why most of those coverage
profile based contig binning methods have to integrate composition profile information
also. Second, on the other hand, if there are a large number of samples, there
will be too many different coverage profiles. We can use more sophisticated
 approach to classify the coverage profiles to reduce the number of bins,
as in those coverage profile based contigs binning methods. Another approach
worthy of note is that there is a method termed partitioning developed in
our group as a divide and conquer approach to scale metagenome assembly. It can be 
considered as a binning approach also, where the reads in the same
partition are more likely to originate from the same microbial organism. We can
try to integrate the partitioning and IGS coverage profile to improve the
accuracy of the binning. In summary, this will be one of the first attempts to
do reads binning. After the IGS/reads binning,  we expect to do better assembly
 and annotation and gain more knowledge about the
function and phylogenetic information.  

We have shown that after adjustment according to sequencing error and collision
rate of the bloom filter, the estimated size of metagenome is close to real
number for synthetic data sets. Howevere, the difference between the estimation and
real number is still increasing with higher error rate, which means there are
other factors that affect the accuracy of estimation. This is worthy of further
investigation. The size estimation of metagenome is extremely important in
metagenomic data analysis and it is closely related to the estimation of
sequencing depth or how much more effort is required to gain enough sequencing
depth. As shown in the results, we are confident that the relative relationship 
between the richness of different samples is reliable from the IGS based alpha
diversity analysis. How accurate the absolute value of the richness or the size
of metagenome in a real data is requires further investigation and new  
statistical model may be needed to adapt to the abundance distribution of IGSs.    
Furthermore, any information about the richness of a sample is beneficial to
the optimal choice of parameter for digital normalization.

Our efforts to examine the effect of sequencing depth on the accuracy of beta
diversity reveals that using a relatively small subset of the whole data set may
get reasonably good result showing the separation of samples after clustering
or ordination. However, how good the separation is seems to be related to the
characteristics of samples and cannot be determined easily before starting the
analysis. Thus, a potential approach should estimate beta analysis in an iterative
way. We already know that using more data will benefit more accurate analysis or better
separation for the purpose of comparing metagenomic samples. In such
iteration procedure, pattern of separation can be monitored as more reads
are loaded into the analysis and the procedure can be stopped as long as the
pattern of the separation of samples is significant enough. This way, we may
save lots of computational cost and still have enough information about the
relationship between samples. 
 
    
    
    
    
    
    

