\chapter{Review of Relevant Literature}

\section{Next-generation sequencing and metagenomics}

\subsection{Next-generation sequencing}
Sequencing technology is changing really fast. Over the past decade, next-generation sequencing(NGS)
 has been the overwhelming technology and almost replaced classic Sanger sequencing technology. 
 Illumina and Roche 454 are the two most popular platforms. Illumina can generate reads with shorter
 length, typically 100 bases for HiSeq and 150 bases for MiSeq \cite{Qin:2010aa, Mason:2012aa},   
 but with lower cost compared to Roche 454 sequencing technology, 
 which generates reads 500 bases to 1K bases longer. In fact a recent study comparing Illumina versus Roche 454 for metagenomics
 shows that both platforms agreed on over 90\% of the assembled contigs and 89\% of the unassembled reads\cite{Luo:2012aa}.
Because of the advantage of low cost, there is an obvious trend that Illumina is dominating the sequencing market, which 
means while designing any tool for metagenomics, the developer should take the relatively short length of Illumina reads
 into account.


\subsection{Metagenomics}
It is believed that the word "metagenomics" was coined in 1998 \cite{Handelsman:1998aa},
 which can be translated as 'beyond the genome' \cite{Gilbert:2011aa}. At that time, basically
 it is the technique of cloning environmental DNA randomly and screening for interesting genes, 
 especially 16S rRNA genes. This technique was firstly applied in practice by Schmidt et al. in 1991 \cite{Schmidt:1991aa}.
 This was a crucial step in the investigation to microbial world. Before that it was standard 
 protocol to culture and isolate microbes from cells or living organisms of any other species and 
 then do analysis. This resulted seriously narrow picture of the diversity of an ecosystem as only a
 small portion of the microbial species (5\% or less) in the biosphere can be cultured with standard culturing techniques \cite{Sogin:2006aa}.
 Metagenomics with the concept of cloning DNA directly from the environment without cultivation 
 brings the researchers the ability to explore the genomic DNA from all the genomes of all the organisms in an environmental
 community, culturable or unculturable.
 
The improvement of next generation sequencing technology with high throughput and low cost has been accelerating
the metagenomics research recently. The number of microbial species in some ecological community is huge. 
In soil it is estimated that there exist millions of species with most of them in low abundance \cite{Gans:2005aa}. 
Only using high throughput next-generation sequencing strategy can it be possible to sample the contents of those populations deeply enough
to cover the rare species.

Currently there are two approaches in metagenomics. One is amplicon metagenomics, to amplify gene of interest like 
16S rRNA genes as taxonomic markers
and sequence the libraries \cite{Sogin:2006aa}, which is the traditional way dating back to 1991 experiment by Schmidt et al. Many classic 
methods to access microbial diversity rely on this approach. The other one is whole genome shotgun metagenomics, to sequence
the libraries of randomly isolated DNA fragments without screening. Since the whole genomes of organisms in the sample are available rather than the limited
 genes of interest like 16S/18S rRNA, this whole genome shotgun sequencing approach can provide better taxonomic resolution and
 more information benefiting other investigation \cite{Tyson:2004aa} \cite{Qin:2010aa}.  Now there have been thousands of metagenomic genomes available
in online database like MG-RAST \cite{Glass:2010aa}.
 
There have been many metagenomics projects focusing on the microbial samples of different kinds of 
habitat, from extreme environment such as acid mine drainage channels with low complexity \cite{Tyson:2004aa}, and medium complexity
 samples like human gut \cite{Qin:2010aa} and cow rumen \cite{Hess:2011aa}, to high complexity samples like seawater \cite{Venter:2004aa} and 
 soil \cite{Gilbert:2010aa}.
 
Metagenomics studies have expanded our knowledge of microbial world in different habitats.
Some of them shed light on the explanation of some serious human deseases. Studies have shown the associations between
human gut metagenomes and type II diabetes \cite{Qin:2012aa}, obesity \cite{Turnbaugh:2009aa, Kau:2011aa} or crohn's disease \cite{Morgan:2012aa}. 

In almost all of these metagenomics projects, diversity analysis plays an important role to supply information about the 
richness of species, the species abundance distribution in a sample or the similarity and difference between different samples, 
all of which are crucial to draw insightful and reliable conclusion. Next the challenges in the assessment of microbial diversity will 
be elaborated.

\section{Challenges in counting k-mers efficiently}
about khmer

\section{Tackling large and error-prone short- read shotgun data sets}
about diginorm


\section{Challenges in measuring diversity of metagenomics}


\subsection{Concept of Diversity}
When we try to characterize an ecological community, diversity measurement is often the first step.
It is always desirable to know how many species there are in a sample, which is the concept of richness and how abundant each
 species is relative to others in the same sample, which is the concept of evenness. 
 They are straightforward conceptually. But in practice, there are a large number of quantities that was suggested
 to measure species diversity to adapt the different scenarios of sampling individuals. 

In a higher level, three diversity indices are well established and used in ecology, $\alpha$-diversity, $\beta$-diversity, and 
$\gamma$-diversity. $\alpha$-diversity is the diversity in one defined habitat or sample. $\beta$-diversity compares species 
diversity between habitats or samples.$\gamma$-diversity is the total diversity over a large region containing multiple ecosystems.

The concept of diversity has two aspects, richness and evenness. Richness is the total number of species
identified in a sample, which is the simplest descriptors of community structure. Evenness is a measure 
of how different the abundance of a species is compared to other species in a community. If all the species in a
 community has the same abundance, the community has a higher evenness diversity. However all natural communities 
 are highly uneven, which means a large number of species has rare abundance in the community. This raises
 a question about the effectiveness of using the measurement of richness to represent diversity. Is a community
 with 1 dominant species and 10 rare species more diverse than a community with 3 dominant species and 2 rare species?
  So a new metrics taking both richness and evenness into account is suggested. Two popular indice are Shannon diversity \cite{shannon2001mathematical},
  which is based on information theory and shows the information in a community as an estimate of diversity, and Simpson diversity index \cite{simpson1949measurement},
  which basically shows the probability that two individuals picked randomly from a community belong to the same species.

Besides these two, Hill \cite{hill1973diversity} proposed a new diversity index to use a weighted counts of species to measure 
diversity, based on the species abundance distribution. This can be considered as a generalized diversity index, 
since both Shannon and Simpson index and richness can be seen as special cases of Hill diversity index.

It is necessary to note that we can not tell if any index is better than other. It all depends on the characteristics
of the community and the process of sampling and other factors. More often an index is used just because it has
been used by many others before in the same scenario, it may be because it is more feasible in this scenario 
but it is not necessarily because it is better or even provide useful information.

In microbial ecology, richness is simply the most popular index to mesure microbial diversity, partially because of 
the challenge raised by the different characteristics of microbial community. Lots of methods to estimate richness in classic 
ecology were borrowed to tackle the problem of estimate microbial diversity, which will be discussed in the next section.

\subsection{Diversity measurement in Microbial Ecology}

There have been numerous mature methods and tools to measure diversity of macroorganisms 
in decades of development of classic ecology. One would think that we just need to borrow 
those methods to use in microbial field. Nevertheless in reality it is not that straightforward. 
The microbial communities are so different from macroorganisms like plant or animal communities,
 with the number of species many order of magnitude larger\cite{Whitman:1998aa}.
This fact raises serious sampling problems. It is really difficult to cover enough fraction of the microbial community
 even with impressively large sample size thanks to modern metagenomic approaches \cite{Roesch:2007aa}.
In a word, diversity measurement is a rather big challenge for microbial communities and novel 
and effective methods are highly demanded \cite{Schloss:2005aa}.

\subsubsection{Concept of Species and OTU Identification using sequence markers}
To borrow the methods of diversity measurement from classic ecology on the use of evaluating
microbial diversity, the first problem is that in microbial world, there is no unambiguous way to define 
"species" \cite{Stackebrandt:2002aa}. It is impossible to identify a microbial individual as a specific species morphologically.
In fact in metagenomics the concept of "species" has been replaced by OTUs(Operational Taxonomic Units). An OTUs are those 
microbial individuals within a certain evolutional distance. Practically we mainly use 16S rRNA genes as the evolutional marker genes, 
because 16S rRNA genes exist universally among different microbial species and their sequences change at a rate corresponding with
the evolutionary distance.  So we can describe microbial individuals with higher than a certain percent(like 97\%) 16S rRNA
 sequence similarity as one OTU, or belonging to one species \cite{Schloss:2005aa}.
 

\subsubsection{Binning of Metagenomic Reads into OTUs}

In classic ecology dealing with samples from macroorganisms communities, before we can
 use any statistical method to measure diversity, it is standard procedure to identify 
 the species of each individual in the sample. It is the same for diversity measurement of 
 microbial communities. Difference is that here we need to place the sequences(individuals) into respective
 "bin" or OTUs(species). There are two strategies to do such binning - Composition-based or intrinsic binning approach 
 and similarity-based or extrinsic binning approach.
 
\paragraph{Composition-based approach}
Lots of efforts have been put to get a comprehensive category of reference microbial genome sequences \cite{HMScience, Wu:2009aa}.
Currently there are a large number of finished or high-quality reference sequences of thousands of microbial species available in 
different databases and this number is still increasing quickly \cite{Markowitz:2012aa, Glass:2010aa, Wang:2007aa}.
So the first intrinsic composition-based approach is to use those reference genomes 
to train a taxonomic classifier and use that classifier to classify the metagenomics reads into bins.
Different statistical approaches like Support Vector Machines \cite{Patil:2012aa}, interpolated Markov models\cite{Brady:2011aa},naive Bayesian classifiers,
and Growing Self Organizing Maps \cite{Rosen:2011aa} were used to train the classifier.
Without using any reference sequences for the training, it is possible to use signatures 
like k-mers or codon-usage to develop reference-independent approach. The assumption is that 
the frequencies distribution of the signatures are similar of the sequences from the same
species. TETRA is such a reference-independent tools using Markov models based on k-mer frequencies \cite{Teeling:2004aa}.
There is another tool using both TETRA and codon usage statistics to classify reads \cite{Tzahor:2009aa}.


\paragraph{Similarity-based approach}
The similarity-based extrinsic approach is to find similarity between the reads sequences and reference sequences and 
a tree can be built using the similarity distance information. MEGAN \cite{Huson:2007aa} is a typical tool using this method,
which reads a BLAST file output. Other sequence alignment tools can also be used here like BowTie2 or BWA. Recently,
an alternative strategy was developed, which only uses the reference sequences with the most information rather than all the reference 
sequences to do alignment. Those reference sequences include 16S rRNA genes or some other specific marker genes. 
The benefit is obvious, it is more time-efficient since there are fewer reference sequences to align to. Also, it can
provide better resolution and binning accuracy since the marker genes can be selected carefully with the best distinguishing power.
AMPHORA2 \cite{Wu:2012aa} and MetaPhlAn \cite{Segata:2012aa} are two typical tools using this strategy.



\subsubsection{Statistics for Diversity Estimation}
After the binning of sequences into OTU, we need statistics to help us estimate the diversity. Many statistical methods have been
developed and widely used in classic ecology to macroorganisms. However the first difference between 
diversity measurement of macroorganisms and microbial community is that generally the microbial community diversity is much larger than 
observed sample diversity, thanks to the high diverse characteristics of microbial community and the limit of 
metagenomics sampling and sequencing. The first approach which is also considered as classic is rarefaction.
Rarefaction curve can be used to compare observed richness among different samples that have been sampled unequally, which
is basically the plot of the number of observed species as a function of the sampled individuals.
It is worth noting that rarefaction curve shows the observed diversity, not the total diversity. We should never forget
those unseen microbial species, which is pretty common for microbial community sampling.  

To estimate the total diversity from observed diversity, different estimators are required.

The first one is extrapolation from accumulation curve. The asymptote of this curve is the total diversity, which means the
number of species will not increase any more with sampling more individuals. To get the value of that asymptote point,
from observed accumulation curve, a function needs to be assumed to fit the curve. Several proposals have been made to use this
extrapolation method \cite{colwell2004interpolating, gotelli2001quantifying}. The problem is that if the sampling effort
only covers a small fraction of the total sample, which means the accumulation curve just starts, it is difficult to find 
an optimal function to fit the curve. Different functions can fit the curve equally well but will deduct dramatically different asymptote
value. So this curve extrapolation method should be used cautiously.

Another one is parametric estimator, which assumes that the relative abundance follows a particular distribution. Then the 
number of unobserved species in the community can be estimated by fitting observed sample data to such abundance
distribution then the total number of species in the community can be estimated.
Lognormal abundance distribution is mostly used in different project since most communities of macroorganisms has a 
lognormal abundance distribution and it is believed that it is also typical for some 
microbial communities \cite{Curtis:2002aa, Schloss:2006aa, Quince:2008aa}. It is understandable that there 
is always controversy as to which models fit the communities best since in an ideal world the abundance 
distribution should be inferred from the data,not be assumed unverifiably. The problem is that we can only 
infer the abundance distribution accurately when the sample size is large enough. There has been some attempts on this direction
recently \cite{Gans:2005aa} and more robust methods are still needed.

If the species abundance distribution can not be inferred, we can still use nonparametric estimators to estimate the 
total diversity without assuming that abundance distribution arbitrarily. These estimators are related to MRR(mark-release-recapture)
 statistics, which compare the number of species observed more than once and the number of species observed only once. 
If current sampling only covers a small fraction of a diverse community, the probability that a species is observed more
 than once will be low and most species will be observed only once. If current sampling is enough to cover most species in
 the community, the opposite will be the case. A series of estimators invented by Chao are the representative estimators
 in this category, including Chao1 \cite{chao1984nonparametric}, Chao2 \cite{Chao:1987aa}, ACE \cite{chao1993stopping} 
 and ICE \cite{lee1994estimating}. For example, Chao1 formular is:\\
 $${S}_{Chao1}={S}_{obs}+\frac{{{n}_{1}}^{2}}{2{n}_{2}}$$
 where ${S}_{obs}$ is the number of species observed, ${n}_{1}$ the number of species observed once(singletons, with only one individule),
 and ${n}_{2}$ the number of species observed twice(doubletons, with exactly two individuals) in the sample. The ACE uses data from
  all species rather than just singletons and doubletons. Its formular is:\\
 $${S}_{ACE}={S}_{abund}+\frac{{S}_{rare}}{{C}_{ACE}}+\frac{{F}_{1}}{{C}_{ACE}}{{\gamma }_{ACE}}^{2}$$
where ${S}_{rare}$ is the number of rare species (with few than 10 individuals observed) and ${S}_{abund}$ is the number of 
abundant species (with more than 10 individuals). 

In past years there are several software packages that have been developed for biodiversity
analysis. Out of them, EstimateS \cite{colwellestimates} is a software that can be used for general purpose diversity analysis,
which implement a rich set of diversity analysis algorithms. However it is not designed specifically
for microbial diversity analysis. So microbial diversity data should be preprocessed to general population data to be fed into
EstimateS. Two other softwares - MOTHUR \cite{Schloss:2009aa} and QIIME \cite{Caporaso:2010aa} are designed for microbial diversity. So they are more popular in 
microbial diversity analysis. CatchAll \cite{Bunge:2011aa} is a relatively newer package, which can estimate
the diversity using both nonparametric and parametric estimators including many variants and return the results 
using different estimators and the respective credibility of the results. 


\subsection{Novel methods required for diversity measurement of large metagenomics sample}

As reviewed in previous sections, the topic of microbial diversity measurement has been investigated 
for a long time with many methods and software packages developed. However there is still lots of 
room for more work to do.

Basically the mainstream methods to measure microbial diversity are still focusing on the use
of 16S rRNA amplicon metagenomics data. Many of the software packages
mentioned above are also supposed to accept 16S rRNA data as input. This is understandable that
after all the concept of OTU is from the similarity of 16S rRNA sequences. Using 16S rRNA data to measure
diversity is popular but is not without problems. 16S rRNAs may not be that reliable to be OTU 
markers. The reliability is sensitive to potential horizontal gene transfer and the variance of gene copy
in bacterium. There is suggestion that alternative marker genes should be used, like single copy housekeeping genes.

%Measurement of microbial diversity is almost always the same as measurement of microbial richness. 
%Lots of efforts have been made to estimate the 

% species - OTU - reads, random 

There are many projects generating whole genome shotgun metagenomics data sets. However they are 
mainly used for assembly and annotation purpose. Less attention was paid to diversity measurement
using these whole genome metagenomics data sets. One possible reason is that the whole genome metagenomics
data sets are often with low depth given the high diversity of metagenomics samples compared to 16S rRNA
ampicon metagenomics data set. Assembly and annotation are always challenging with the low depth and lack of 
reference sequences. It is also true for diversity measurement. On the other hand, although with low depth, some whole genome metagenomics 
data sets are with large size because of the high diversity. For instance, there may be 4 petabase
pairs of DNA in a gram of soil\cite{Zarraonaindia:2013aa}. Many of those methods for sequence binning or diversity 
estimation do not scale well and will not work for large metagenomics data sets. For instance,
many composition-based binning approach involves k-mer/signature frequency distribution calculation, which is 
rather computationally expensive. Even basic sequence alignment will be impossible for large metagenomics data set.
Many of those statistical software packages to estimate diversity using various estimators are not prepared 
for the large scale of whole genome metagenomics data. 

With the development of next generation sequencing technology, the cost of sequencing is dropping rapidly. 
Whole genome metagenomics sequencing is more popular and large amount of metagenomics data is 
being generated with increasing speed, which can not be even met by the increase of computational capacity.
Novel methods that can scale well are extremely needed to deal with the increasingly large metagenomics data
set. 


