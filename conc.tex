%check the reads coverage in the sample where it is from
%
%\section{de bruijn graph mapping}
%
%check the reads coverage in another sample
%
%\section{IGS-based diversity analysis}
%


\chapter{Conclusion}

%Advantage:

%not only HMP, high depth, data
%but also low depth data, like soil metagenomics, which is impossible to use traditional method 


We have developed a novel statistical framework for enabling microbial diversty
 analysis using whole genome shotgun metagenomic reads data set without the
requirement of assembly, binning, reference or annotation. This dissertation
covers an overview of existing approaches of doing microbial diversity analysis
 of metagenomic samples, especially based on the concept of OTU, including the
steps in the procedure, like metagenomic assembly, contig binning, statistical
analysis of OTU abundance information to estimate the microbial diversity. Next
the statistical framwork based on the novel concept of IGS was discussed. As
the foundation of the framework, we described a novel method to count k-mers
efficiently and a scalable approache to retrieve the coverage of a read in a 
data set based on efficient and online k-mer counting. We also introduced the
application of this approach in reducing the redundancy of metagenomic reads
dataset, which is beneficialf to other tasks in metagenomic data analysis, like
assembly or error trimming. Finally, we discussed how we developed the concept 
of IGS based on the method of efficient k-mer counting and digital
normalization discussed before.  
The application of IGS to analyze microbial diversity of metagenomic data sets 
was discussed and the performance of the IGS method on simulated data sets and
real data sets were demonstrated in the final chapters. In this chapter, we
summarize how the novel statistical framwork based on IGS makes a difference to
 the diversity analysis in current microbial ecology research. 
Finally the possible avenues for future work will be discussed.

  

\section{Summary}

Diversity analysis is a key part of the microbial ecological research, like of
macroorganism ecology. However due to the obscure definition of the term
"species" in microbial ecology, we can virtually never measure the diversity of
species directly, rather we use other taxonomic concepts like operational
taxonomic unit (OTU) to evaluate the diversity of microbial community, instead
of species. rRNA sequencing reads may be classified into different OTUs.
Shotgun whole genome sequencing reads can also be classified into OTUs. But
most if not all
the existing methods based on the concept of OTU rely on some kinds of
preprocessing of original reads data like assembly or some kinds of external 
information like reference sequences for annotation. Both of the prerequisites
are not satisfied for many metagenomic project. For some low coverage
metagenomic reads data, assembly can only extract partial information from the
species with higher sequencing depth. It is common only a small proportion of
reads can be used in assembly. The reference sequence database is far from
completion enough especially for environmental sample from soil or sea water.
So applying these methods can only yield a partial picture of diversity of
the microbial community the metagenomic sequencing data covers. 

The IGS based
method discussed in this dissertation offers a brand new framework that
bypasses the difficult tasks like assembly, binning, or annotation whthout the
requirement of reference sequence database. It can take advanrage of all the 
information in the metagenomic reads data and gain a full picture of the
diversity of the microbial community. The fact that this is a whole new
framework with the concept of IGS to replace OTU as the taxinomic unit to
analyze diversity means that this framework can be used to do all the possible
diversity analysis that OTU-based framework can do. This is a more thorough
solution than many other methods that are developed to solve some specific
problems in the field of diversity analysis. For example, there are several
methods developed to estimate the richness of a metagenomic sample or the size 
of metagenome (cite nonparel,etc.) Our IGS based framework can not only
estimate the richness or size of metagenome as shown in the section above, but
also it can estimate the evenness or species abundance distribution of a
metagenomic sample, which is also an important aspect of alpha diversity
analysis. For beta diversity or compositional similarity analysis between
metagenomic samples, there are several methods developed to compare metagenomic
samples based on reads mapping or counting shared reads. (cite compareads)
But basically they only estimate abundance-based similarity, similar to
the Bray-Curtis indices used in the experiment discussed in the section above.
Actually what is not shown in the result in the section above is that the 
IGS-based framework can also be used to estimate incidence-based similarity, if
such information is interesting to the researcher.

Besides the advantage of the IGS based framework that it has a broad
application with its versatile potential, this freamework is also efficient and
highly scalable enough to handle extremely large metagenomic sequencing data
sets. We have discussed the efficiency of the novel k-mer counting method and
the following method of digital normalization, with the ability to retrieve the
coverage of a read accurately and efficiently. We also did thorough analysis
to the effect of the size of used data structure to the accuracy. We can take
advantage of the probabilitic characteristics of the data structure to make a
trade-off between expected analysis accuracy and expected usage of
computational power. In this way, we make the analysis highly scalable to
the increasing size of metagenomic sequencing data.  

Also, for the purpose of estimating microbial diversity of metagenomic samples,
we analyzed the effect of sequencing depth to the accuracy of the estimation.
It is obvious and expected that using more reads data with higher sequencing
depth increases the accuracy of such diversity estimation. Especially for beta
diversity or similarity analysis between samples, reads data with relatively
low sequencing depth can still yield decent result, as shown in the experiment
with synthetic data and real soil data set. For alpha diversity like richness
estimation, using reads data with lower sequencing depth gets the estimation
farther from the real number, but although the absolute value of such species
richness of a sample is not that accurate, the relative relationship of species
richness between samples are less prone to smaller reads data with lower
sequencing depth. These analysis and observation convince that for specific
purpose, only a subset of the large metagenomic reads data can be enough to
achieve reasonably satisfying result. Under certain circumstance, this feature
is quite helpful and can reduce the computational expense dramatically. 


\section{Future Work}


Though this dissertation detailed a new approach to analyze microbial diversity
using whole genome shotgun sequencing data without the requirement of assembly,
binning, or annotation, there is still plenty of room for improvement.  

Basically we question the IGS based method is developed to answer is how many
species there are in the sample or how similar the samples are between each
other, mostly focusing on the quantitative aspect. Admittedy these are 
important questions the microbial ecologists want to answer, they are also
curious to the qualitative questions, like if two samples are different, what
are the stuffs that attribute to the diffence,and what are the function of
those stuffs.cite rob knight's new paper.  A natural expansion of the IGS based framework will lead to
answer questions like this. 

Now we have an efficient and scalable approach to get the coverage of a read in
a sample, it is straightforward to extract the reads according to its coverage
profile across samples so we can get a subset of reads that have specific
characteristics, like the reads are common to all the samples. In this way we
may collect these "common" reads from samples together and try to co-assemble
them since now they should have higher coverage. Or we can get a subset of
reads that are common in a group of samples but do not exist in another group
of samples, like the samples from patients and healthy persons. These
"signiture" reads may offer important insights to understand what happens
to the microbial community while the environment changes. Admittedly these
kinds of "extraction" can be implimented using other methods like reads 
alignment method. But they may not be as efficient and scalable as the IGS
based method, especially for extremely large metagenomic data thanks to sliding
cost of sequencing.


One advantage of the IGS based method is that binning is not required in this
procedure. Firstly traditionally binning is used to classify contigs after some
kinds of assembly effort. The similarity based binning method relies on
sequence alignment, which is not efficient, especially infeasible for large
metagenomic data. Also, reference sequences normally are required for
similarity based binning approach. The composition-based approach relies on the
frequency profile of sequence signatures and machine learning approach on that
profile, which is computationally expensive. As reviewed in the second chapter,
the third approach based on coverage profile across samples was developed
recently. (cite concoct, groopm, etc.), although mostly often, the coverage
profile is used with the companion of composition frequency profile to classify
contigs. The assumption on which the coverage profile based binning approaches
are based on, that contigs with similar coverage profile across samples are
more likely to be from the same microbial organism, is actually similar to the
assumption on which using IGS to do beta diversity is based, that the IGSs with
similar coverage profile across samples are likely to be frome the same
microbial organism. It is promising to classify the IGSs by the coverage
profile across samples. We have already overcome the challenge of retrieving the
coverage profile efficiently based on the probabilistic data structure, while
in those coverage profile based binning approaches the coverage profile is 
normally retrieved by assembly of contigs and mapping
reads back to contigs, which both require higher coverage reads to do assembly
and are computationally expensive. 

There are two problems to deal with about this coverage profile based
IGS binning approach. First, with relatively small number of samples, the
resolution will be limited, since the total number of different coverage
profiles will be limited. This is probably the reason most of those coverage
profile based contig binning method integrate composition profile information
also. Second, on the other hand, if there are large number of samples, there
will be too many diferent coverage profiles. We can use more sophisticated
clustering approach to the coverage profiles to reduce the number of bins,
as in those coverage profile based contig binning methods. Another approach
worthy of note is that there is a method termed partitioning developed in
our lab as a divide and conque approach to scale metagenome assembly. It can be 
considered as a rough binning approach also, where the reads in the same
partition are more likely to originate from the same microbial organism. We can
try to integrate the paritioning and IGS coverage profile to improve the
accuracy of the binning. In summary, this will be one of the first attempts to
do reads binning. After the IGS/reads binning,  we expect to do better assembly
 and annotation expected to be more accurate to gain more knowledge about the
function and phylogenetical information.  

As shown in Figure, after adjustment according to sequencing error and collison
rate of the bloom filter, the estimated size of metagenome is close to real
number for synthetic data sets. But the difference between the estimation and
real number is still increasing with higher error rate, which means there is
other factor that affect the accuracy of estimaiton. This is worthy of further
investigation. This kind of estimation of metagenome is extremely important in
metagenomic data analysis and it is closely related to the estimation of
sequencing depth or how much more effort is required to gain enough sequencing
depth. As shown in the result, we are confident that the relative relationship 
between the richness of different samples are reliable from the IGS based alpha
diversity analysis. How accurate the absolute value of the richness or the size
of metagenome in a real data requires further research and new  
statistical model may be needed to adapt to the abundance distribution of IGSs.    
Furthermore, any information about the richness of a sample is beneficial to
the optimal choice of parameter for digital normalization.

The analysis of the effect of sequencing depth to the accuracy of beta
diversity shows that using a relatively small subset of the whole data set may
get reasonably good result showing the separation of samples after clustering
or ordination. However how good the seperation is seems to be related to the
characteristic of samples and can not be determined easily before starting the
analysis. So a potential approach is to do such beta analysis in an iterative
fasion. We already know that using more data will yield better result or better
separation for the purpose of comparing metagenomic samples. So in such
iteration procedure, how good the separation is can be monitored as more reads
are loaded into the analysis and the procedure can be stopped as long as the
pattern of the separation of samples is significant enough. This way, we may
save lots of computational expense and still have enough information about the
relationship between samples. 
 
    
    
    
    
    
    

